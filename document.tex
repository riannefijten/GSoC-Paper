\documentclass{bioinfo}
\copyrightyear{2005}
\pubyear{2005}

\begin{document}
\firstpage{1}

\title[short Title]{A Pathvisio plugin for visualization of metabolite information}
\author[Sample \textit{et~al}]{R. Fijten\,$^{1,*}$, Egon Wilighagen\,$^{1}$ \footnote{to whom correspondence should be addressed}}
\address{$^{1}$Department of Bioinformatics, Maastricht University, Address XXXX}

\history{Received on XXXXX; revised on XXXXX; accepted on XXXXX}

\editor{Associate Editor: XXXXXXX}

\maketitle

\begin{abstract}

\section{Motivation:}
Recent developments have shown the importance of metabolomics in human systems biology research.  Given their crucial and central role as particles capturing information from all functional levels of the cell, they can provide the link between transcription and actual phenotyping. It is therefore important to develop tools that implement metabolomics data into for example pathway development tools such as Pathvisio. Pathvisio is an open source pathway development program that utilizes pathways in the WikiPathways database. A high number of these pathways contain metabolites that play a crucial role in that pathway. The plugin for metabolite information can be used to access data for these metabolites directly from within the program without using a browser. The plugin visualizes widely used IDs, molecular formula's, SMILES, InChIs and implements NMR and MS data. 

\section{Results:}
Use-case: When selecting a metabolite in pathvisio, the plugin will show a number of types of data, including predicted Mass spectrometry and Nuclear Magnetic Resonance data.

\section{Availability:}
The plugin is available on the  \href{http://pathvisio.org/wiki/PluginDocumentation}{plugin page} of the Pathvisio website
\section{Contact:} Rianne Fijten: \href{riannefijten@gmail.com}{E-mail}, \href{https://twitter.com/riannefijten}{Twitter}
\end{abstract}

\section{Introduction}
Wikipathways: what is it? How does it work?

Pathvisio: why use pathvisio?

metabolomics: Why is it important?

Why develop a plugin for metabolite information

\section{Approach}


\begin{methods}
\section{Methods}

What does the plugin provide in general?

Screenshot

What are the individual parts?

General information

NMR data

MS data

\end{methods}

\begin{figure}[!tpb]%figure1
%\centerline{\includegraphics{fig01.eps}}
\caption{Caption, caption.}\label{fig:01}
\end{figure}

\begin{figure}[!tpb]%figure2
%\centerline{\includegraphics{fig02.eps}}
\caption{Caption, caption.}\label{fig:02}
\end{figure}

\section{Discussion}




%%%%%%%%%%%%%%%%%%%%%%%%%%%%%%%%%%%%%%%%%%%%%%%%%%%%%%%%%%%%%%%%%%%%%%%%%%%%%%%%%%%%%
%
%     please remove the " % " symbol from \centerline{\includegraphics{fig01.eps}}
%     as it may ignore the figures.
%
%%%%%%%%%%%%%%%%%%%%%%%%%%%%%%%%%%%%%%%%%%%%%%%%%%%%%%%%%%%%%%%%%%%%%%%%%%%%%%%%%%%%%%






\section{Conclusion}

Summary of the plugin and what it can do.

Future development



\section*{Acknowledgement}
Text Text Text Text Text Text  Text Text.  \citealp{Boffelli03} might want to know about  text text text text

\paragraph{Funding\textcolon} Text Text Text Text Text Text  Text Text.

\bibliographystyle{natbib}
%\bibliographystyle{achemnat}
%\bibliographystyle{plainnat}
%\bibliographystyle{abbrv}
%\bibliographystyle{bioinformatics}
%
%\bibliographystyle{plain}
%
\bibliography{document}

\end{document}
